\documentclass[a4paper,landscape,columns=5]{cheatsheet}

\usetikzlibrary{quotes,angles,decorations.pathmorphing}

%Multi-line comment
\newcommand{\comment}[1]{}

%Title info
\title{\textsc{\mbox{PH1012 Help Sheet}}}
\author{Daniel Laing}
\date{\today}

%Info for PDF
\pdfinfo{
	/Title (PH1012 Formulae Sheet.pfd)
	/Author (Daniel Laing)}
	
%Set margins
\geometry{top=1cm,left=1cm,right=1cm,bottom=1cm}	

%Set column spacing
\setlength{\premulticols}{1pt}
\setlength{\postmulticols}{1pt}
\setlength{\multicolsep}{1pt}
\setlength{\columnsep}{4pt}

%Create frame styles
\mdfdefinestyle{normal}{linecolor=gray,linewidth=1pt,%
	leftmargin=0mm,rightmargin=0mm,skipbelow=0mm,skipabove=0mm}

\begin{document}
\maketitle\vspace{5pt}
\small
\usection{Quantum Phenomena}
%%{\color{gray}\rule{\linewidth}{1pt}}
\subsection{Quantisation \(\left(n\!\in\!\N\right)\)}	
\begin{tabular}{lr}
	\mbox{Energy of photon (\(\gamma\)):}
	 & \hspace{-10pt}{\color{eqbl}\multirow{2}{*}{
		\(\begin{aligned}
		E_{\gamma} &= nh\nu\\
		l &= n\hbar
		\end{aligned}\)}}\vspace{3pt}\\
	\mbox{Angular momentum:} & 
\end{tabular}

\subsection{Particle Nature of Light}
	{\setstretch{0.8}
		\begin{enumerate}
		\item Photoelectric Effect:\\\(\to\) light is absorbed in "chunks".
		\item X-ray Production:\\\(\to\) light is produced in "chunks".
		\item Compton Scattering:\\\(\to\) light has momentum.
		\end{enumerate}
	}

\subsection{Photoelectric Effect\quad\((L2)\)}
	Classical: \(\to\) electron receives oscillating force, magnitude increases until ejected.\vspace{-3pt}
	{\color{gray}\rule{\linewidth}{1pt}}
	Quantum: \(\to\) electron receives kick from \(\gamma\), energy transferred. Energy breaks attraction (work function, \(W\)):\vspace{-3pt}
	\[\begin{aligned}
		\hspace{-30pt}E_k &= h\nu-W,\quad eV_0=E_{kmax}\\
		\hspace{-30pt}\implies eV_0 &= h\nu -W
	\end{aligned}\]\vspace{-10pt}

\subsection{X-ray Production\quad\((L3)\)}
Produced by ``breaking energy'' of electron passing nucleus.

Inner shell interactions give characteristic radiation (\(K_{\alpha}\) and \(K_{\beta}\)).

\(e^-\) completely stopped gives \(\lambda_{min}\):
	\[\to\lambda_{min}=\dfrac{hc}{eV}\]\vspace{-13pt}

\subsection{Compton Scattering\quad\((L4)\)}
\(\gamma\) incident on electron scatters:

	\quad \(\to\) assumes \(\gamma\) has momentum.\vspace{-5pt}
{\color{gray}\rule{\linewidth}{1pt}}

But see 2 peaks, unchanged \& shifted, shift depends on angle:
\mbox{{\color{eqbl}\(\to\Delta\lambda=\dfrac{h}{mc}\left(1-\cos\vartheta\right)\)\quad{\tiny\(\left(=\dfrac{2h}{mc}\sin^2\dfrac{\vartheta}{2}\right)\)}}}

\(\lambda\) shift from scattering \(e^-\).

Unchanged \(\lambda\) from strongly bound \(e^-\), effectively interacting with atom:

	\quad\ \(\to\,\approx\!\infty\) mass \(\implies\Delta\lambda\to 0\)

\subsection{Matter Waves\quad\((L5)\)}
Double slit with single \(\gamma\):

	\quad \(\to\gamma\) interferes with itself.

Can't predict landing place, only probability, \(P\propto |E|^2\) Position not defined until measurement.

All matter has wave-like properties:
\vspace{-8pt}\[\to\lambda=\dfrac{h}{p}\]\vspace{-20pt}

\subsubsection{\textbf{\quad Bragg Scattering}}
Needs:
\begin{enumerate}
	\item \(\phi\approx 2\vartheta\)\hspace{1cm}
		\begin{tikzpicture}[transform canvas={scale=0.65}]
			\draw[blue] (0,0) coordinate (a)
			-- (4,0) coordinate (b);
			\draw[blue] (0,0.5) coordinate (c)
			-- (4,0.5) coordinate (d);
			\draw[->] (0,1) coordinate (e)
			-- (2,0) coordinate (o);
			\draw[->] (o)
			-- (4,1) coordinate (f);
			\draw[->] (0,1.5) coordinate (g)
			-- (2,0.5) coordinate (o');
			\draw[->] (o')
			-- (4,1.5) coordinate (h);
			\coordinate (x) at (3.75,0);
			\coordinate (y) at (3.75,0.5);
			\draw[<->] (x) -- (y) node[below right] {\(d\)};
			\draw pic[%
				"\(\vartheta\)",
				draw=orange,
				-,
				angle eccentricity=0.8,
				angle radius=25pt,
			] {angle=g--o'--c};
			\draw pic[%
				"\(\vartheta\)",
				draw=orange,
				-,
				angle eccentricity=0.8,
				angle radius=25pt,
			] {angle=d--o'--h};
			\draw pic[%
				"\(\phi\)",
				draw=orange,
				-,
				angle eccentricity=0.6,
				angle radius=15pt,
			] {angle=h--o'--g};
		\end{tikzpicture}
	\item \(\lambda\approx\) atomic size \(\left(d\right)\).
	\item Moderate acceleration \(\left(V\approx 50V\right)\).
\end{enumerate}
\[2d\sin\vartheta=n\lambda\hspace{12pt}n\!\in\!\N\]
Larger particles also interfere

	\quad e.g. small viruses, vitamins.

Accelerated \(e^-\) have energy:
	\vspace{-8pt}\[\to E_{e^-}=\dfrac{h}{\sqrt{2m_eeV}}\]\vspace{-17pt}

\subsection{Uncertainty Principle\quad\((L6)\)}
\[\Delta x\Delta p_x\geqslant\dfrac{\hbar}{2\pi}\]

%------
\[E=\dfrac{h}{\sqrt{2m_eeV}}\]\\
%------
\[
\dfrac{d^2\Psi(x)}{dx^2}+\dfrac{2m}{\hbar^2}\left[E-V(x)\right]\Psi(x)=0
\]\\
\[
\textrm{Reduced mass: }\mu=\dfrac{m_1+m_2}{m_1m_2}
\]\\
\[
E=\dfrac{h^2n^2}{8mL^2}\hspace{12pt}n\!\in\!\N
\]\\
\[
\dfrac{1}{\lambda}=R_\infty Z^2\left(\dfrac{1}{n_f^2}-\dfrac{1}{n_i^2}\right),\hspace{12pt}R_\infty=\dfrac{1}{(4\pi\varepsilon_0)^2}\cdot\dfrac{me^4}{4\pi\hbar^3c}
\]\\
\[
E_n=-\dfrac{Z^2}{n^2}R_y,\enspace R_y=hcR_\infty
\]\\
\[
r_n=a_0\dfrac{n^2}{Z},\hspace{12pt}a_0=\dfrac{4\pi\varepsilon_0\hbar^2}{me^2}
\]

\newpage
\usection{Mechanics}

\[
\alpha=\dfrac{d\omega}{dt}=\dfrac{d^2\vartheta}{dt^2}
\]\\
\[
v=r\omega\left(=r\dfrac{d\vartheta}{dt}\right),\qquad a=r\alpha\left(=r\dfrac{d\omega}{dt}\right)
\]\\
\[
F_{central}=\dfrac{mv^2}{r}=mr\omega^2
\]\\
\[
|\vec{F}|=G\dfrac{m_1m_2}{r^2}
\]\\
\[
g(r)=\dfrac{GM(r)}{r^2}
\]\\
\[
P^2=\dfrac{4\pi^2}{GM}a^3
\]\\
\[
U(r)=-G\dfrac{Mm}{r}
\]

\newpage
\usection{Lasers}
\[
\Delta\nu = \frac{c\Delta\lambda}{\lambda ^2}
\]\\
\[
R = \frac{A_{21}}{\rho (\nu)B_{21}} = \exp\left[{\dfrac{h\nu}{k_BT}}\right]-1
\]\\
\[
I_2=I_1 e^{2 \alpha L} R_1 R_2
\]\\
\[
\textrm{Amp. threshold: } e^{2 \alpha L} R_1 R_2 = 1
\]\\
\[
\textrm{Max. efficiency: }\eta=\dfrac{\nu_{emission}}{\nu_{pump}}=\dfrac{\lambda_{pump}}{\lambda_{lasing}}
\]\\
\[
2L_{optical}=m\lambda_m\hspace{12pt}m\!\in\mathbb{N}
\]\\
\[
\textrm{Reflectivity, }R=\left(\dfrac{n_1-n_0}{n_1+n_0}\right)^{\!2}
\]\\
\[
\Delta \lambda_{FSR} =\dfrac{\lambda^2}{2L_{opt}} 
\]\\
\[
P_{peak}=\dfrac{E_{pulse}}{t_{pulse}}
\]\\
\[
P_{average}=P\!RF\!\cdot\!E_{pulse}
\]\\
\[
t_{pulse}\approx\dfrac{2L_{opt}}{c(1-Re\!f)}
\]\\
\[
t_{interval}=\dfrac{2L_{opt}}{c}
\]

\end{document}
